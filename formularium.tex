\documentclass[a4paper,kul]{kulakarticle} %options: kul or kulak (default)

\usepackage[utf8]{inputenc}
\usepackage[dutch]{babel}

\date{Academiejaar 2024 -- 2025}
\address{
	Industriële Ingenieurswetenschappen \\
	Chemie \\
	Arn Mignon}
\title{\href{https://github.com/TimoNotThy/Chemie-formularium}{Formularium}}
\author{\href{https://github.com/TimoNotThy}{Timo Vandevenne}}
\usepackage{hyperref}
\usepackage{graphicx}
\usepackage{amsmath, amssymb, amsthm}
\usepackage{siunitx}
\usepackage{flafter} 
\usepackage{pdfpages}
\usepackage{caption}
\usepackage{subcaption}
\usepackage{titlesec}
\usepackage[shortlabels]{enumitem}
\usepackage{gensymb}
\usepackage{upgreek}
\usepackage{bm}

\setcounter{secnumdepth}{4}

\titleformat{\paragraph}
{\normalfont\normalsize\bfseries}{\theparagraph}{1em}{}
\titlespacing*{\paragraph}
{0pt}{3.25ex plus 1ex minus .2ex}{1.5ex plus .2ex}

\hypersetup{
	pdftitle={Formularium Chemie},
	pdfsubject={},
	pdfauthor={Timo Vandevenne},
	pdfkeywords={}
}
\newcommand{\varitem}[2]{\textbf{\(\mathbf{#1}\)} #2}

\begin{document}
\maketitle
\newline
\\ % Verwijder volgende zin op het einde van het semester!
Dit document is nog niet klaar, als we nieuwe formules zien zal ik deze toevoegen.


\begin{center}
	\begin{tabular}{>{$}l<{$} | p{0.55\textwidth}} % Left column for formula, right for variables and explanations
		\textbf{Formule} & \textbf{Variabelen en uitleg} \\
		\hline
		\text{Verdunningsregel: } M_i V_i = M_f V_f
		& \varitem{M}{Molariteit [mol/l]} \\
		& \varitem{m}{Molaliteit [mol/kg]} \\
		PV=nRT
		& \varitem{P}{Druk} \\
		& \varitem{V}{Volume} \\
		& \varitem{R}{Gasconstante} \\
		& \varitem{T}{Temperatuur [K]} \\
		
		\hline% \textbf{Thermochemie} \\
		
		\Delta\text{U}=q+\text{w} 
		& \varitem{\Delta U}{Verandering van interne energie} \\
		& \varitem{q}{warmteuitwisseling met omgeving \newline (q>0: warmte van omgeving in systeem)} \\
		& \varitem{w}{Arbeid verricht op/door het systeem \newline (w>0: arbeid op systeem)} \\
		
		\text{w}=-P\Delta V 
		& \varitem{\Delta V}{Volumeverandering} \\
		
		\text{Wet van Hess:}
		& \varitem{\Delta H^0_{rxn}}{Reactieenthalpie} \\
		\Delta H^0_{rxn}\! =\! \sum i\Delta H^0_f(prod.)\! -\! \sum j\Delta H^0_f(reag.) 
		& ($\Delta \text{H}^0_{rxn}$>0: endotherme reactie) \\ %Uitleg bij vorige maar appart gezet voor layout purposes
		& \varitem{H^0_f}{Standaardvormingsenthalpie} \\
		& \varitem{i, j}{coefficiënten in reactievergelijking} \\
		
		q=ms \Delta T 
		& \varitem{m}{massa [g]} \\
		& \varitem{s}{Specifieke warmte [$\frac{J}{g \degree C}$]} \\
		
		
		q=C \Delta T 
		& \varitem{\Delta T}{Temperatuurverandering} \\
		& \varitem{C}{Warmtecapaciteit} \\
		q_{sys}=0 \Leftrightarrow q_{rxn}+q_{cal}+q_{opl}=0 \\
		q_{rxn}=n\Delta H^0_{rxn} \\
		%Kwantumtheorie
		
		\hline
		
		E=h \upnu = h \cfrac{c}{\lambda} 
		& \varitem{E}{Energie [J]} \\
		& \varitem{h}{constante van Planck = $6.62 \cdot 10^{-34}$Js} \\
		& \varitem{\bm{\upnu}}{frequentie [Hz]} \\
		& \varitem{c}{Lichtsnelheid = $3 \cdot 10^8 \frac{m}{s}$} \\
		& \varitem{\bm{\lambda}}{Golflengte [m]} \\
		
		E_{kin,e^-}=h \upnu - W &
		\varitem{W}{Werkfunctie: maat voor hoe sterk $e^{-}$ in metaal worden vastgehouden} \\
		
		\text{De Broglie: }
		\lambda=\frac{h}{p} = \frac{h}{mu}
		& \varitem{p}{Impuls [$\frac{kg \cdot m}{s}$]} \\
		& \varitem{m}{Massa bewegend deeltje [kg]} \\
		& \varitem{u}{Snelheid} \\
		
		\hline
		
		
		\text{Wet van Dalton: }P_i=y_iP_{tot}
		& \varitem{P_i}{Partieeldruk} \\
		& \varitem{y_i}{Molfractie \textbf{gas} [\%]} \\
		
		\text{Wet van Raoult: }P_i=x_iP_i^0
		& \varitem{x_i}{Molfractie \textbf{vloeistof} [\%]} \\
		& \varitem{P^0_i}{Dampdruk} \\
		
		\text{Wet van Henry: }P_i=x_iH_i=\cfrac{C_i}{k}
		& \varitem{C_i}{Concentratie} \\
		& \varitem{H_i}{Henry constante} \\
		& \varitem{k}{gegeven constante bij bep. temp} \\
	
	\end{tabular}
	\newpage
	\begin{tabular}{>{$}l<{$} | p{0.55\textwidth}}		
		\hline
		%colligatieve eigenschappen
		\Delta T_b=iK_bm
		& \varitem{\Delta T_b}{Kookpuntsverhoging} \\
		\Delta T_f=iK_fm
		& \varitem{\Delta T_f}{Vriespuntsverlaging} \\
		& \varitem{i}{Van 't Hoff factor: aantal opgeloste deeltjes waarin een verbinding voorkomt in oplossing} \\
		& \varitem{K_b, K_f}{karakteristiek van het oplosmiddel} \\
		& \varitem{m}{Molaliteit [mol/kg]} \\
		
		\pi = i M\! RT
		& \varitem{\bm{\pi}}{Osmotische druk} \\
		\Delta P=x_{\text{opgeloste stof}}\ P^0_{\text{oplosmiddel}}
		& \varitem{\Delta P}{Dampdrukverlaging} \\
		%& \varitem{P^0}{Druk} \\
		
		%Kinetica
		\hline
		v=k[A]^x[B]^y
		& aA+bB $\rightleftharpoons$ cC+dD \\
		& \varitem{v}{Reactiesnelheid [M/s]} \\
		& \varitem{k}{Snelheidsconstante [Eenheid afh. van reactieorde]} \\
		
		\text{Arrhenius vergelijking:}
		& \textbf{x}=a, \textbf{y}=b indien elementaire stap \\
		k=Ae^{\frac{-\! Ea}{RT}}
		& \varitem{E_a}{Activeringsenergie [kJ/mol]} \\
		
		%Chemisch evenwicht
		\hline
		K=\dfrac{[C]^c[D]^d}{[A]^a[B]^b}
		& \varitem{K}{Evenwichtsconstante (K>1: Evenwicht naar rechts)} \\
		& \varitem{\bm{[X]}}{Concentratie van stof X [M]\! =\! [mol/l]} \\
		& \varitem{Q}{Reactieconstante, K met actuele concentraties} \\
		& (Q>K: systeem naar links voor evenwicht) \\
		
			%Niet echt een formule maar wel belangrijk
			\text{Principe van Le Châtelier}
			& Systeem compenseert uitwendige stress gedeeltelijk \\
			& $\bullet$ Concentratieverandering \\
			& $\bullet$ Druk \& volumeverandering \\
			& $\bullet$ Temperatuursverandering $\rightarrow$ K verandert \\
			& $\bullet$ Katalysator \& inert gas hebben geen invloed \\
		
		%Zuren en basen
		\hline
		pH=-\log[H^+]=-\log[H_3O^+] \\
		pOH=-\log[OH^-]=14-pH\\
		K_a=\cfrac{[H^+][A^-]}{[HA]}
		& \varitem{K_a}{Aciditeitsconstante (\bm{$pK_a$}=$-\log K_a$)} \\K_b=\cfrac{[OH^-][B^+]}{[B]}
		& \varitem{K_b}{Basiciteitsconstante (\bm{$pK_b$}=$-\log K_b$)} \\
		K_a K_b=K_w
		& \varitem{K_w}{Dissociatieconstante van water}\\
		pK_a+pK_b=pK_w
		& $\text{K}_w=[H^+][OH^-]=10^{-14}$ bij 25\degree C \\
		
		%Zuur-base evenwichten en oplosbaarheid
		K_{sp}=[C]^d [D]^d
		& \varitem{K_{sp}}{Oplosbaarheidsproduct: beschrijf het oplossen van een ionische verbinding in water} \\
		Q=[C]^c_0 [D]^d_0
		& \varitem{Q}{Reactiequotiënt, K$_{sp}$ met actuele concentraties} \\
		& \varitem{\bm{[X]_0}}{Concentratie voor reactie} \\
		& $\bullet$ Q<K$_{sp}$: Onverzadigde oplossing   $\rightarrow$ Geen neerslag \\
		& $\bullet$ Q<K$_{sp}$: Verzadigde oplossing     $\rightarrow$ Net geen neerslag \\
		& $\bullet$ Q<K$_{sp}$: Oververzadigde oplossing $\rightarrow$ Neerslag onstaat \\
		
		%Redoxreacties en elektrochemie
		\hline
		\text{Nernst: } E=E^0 - \cfrac{RT}{nF} \log Q
		& \varitem{F}{Faraday constante: lading 1 mol e$^-$} \\
		E^0_{cel}=\cfrac{RT}{nF} \log K
		& \varitem{E^0_{cel}}{Celpotentiaal ($E^0_{cel}$>1: Formatie producten)} \\
		E^0_{cel}=E^0_{ox}+E^0_{red}=E^0_{red, anode}+E^0_{red, kathode}
	\end{tabular}
\end{center}

\end{document}